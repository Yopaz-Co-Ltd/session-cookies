\documentclass[aspectratio=169]{beamer}
\usetheme{CambridgeUS}
\usepackage[utf8]{inputenc}
\usepackage[vietnamese]{babel}
\usepackage{graphicx}

\title{Cookie \& Session\\Bạn đã thật sự hiểu?}
\author{Lê Văn Quân}
\date{\today}

\begin{document}

\begin{frame}
  \titlepage
\end{frame}

\begin{frame}{Mở đầu câu chuyện}
  \begin{itemize}
    \item Ngày xưa, khi chưa có cookie hay session
    \item Server không thể phân biệt được người dùng nào đang truy cập.
    \item Lý do:
    \begin{itemize}
        \item HTTP là giao thức \textbf{stateless}
        \item Mỗi request độc lập, không ``nhớ'' request trước.
    \end{itemize}
  \end{itemize}
\end{frame}

\begin{frame}{Khi không có cookie/session, server làm gì?}
  \begin{enumerate}
    \item Gắn ID vào URL (URL rewriting):\\
    \texttt{/profile?user\_id=123}
    \item Dùng hidden field trong form:\\
    \texttt{<input type="hidden" value="123" name="user\_id">}
    \item Theo dõi địa chỉ IP
  \end{enumerate}
  \vspace{1em}
  \textbf{Nhược điểm:} không bảo mật, dễ giả mạo, không tiện lợi
\end{frame}

\begin{frame}{Sự ra đời của Cookie}
  \begin{itemize}
    \item \textbf{1994} – Lou Montulli (kỹ sư tại Netscape) đề xuất cơ chế Cookie
    \item Mục tiêu: giải quyết việc lưu trạng thái người dùng trong giao thức HTTP (vốn stateless)
    \item Cookie cho phép lưu một lượng nhỏ dữ liệu trên trình duyệt:
    \begin{itemize}
      \item ID người dùng
      \item Tuỳ chọn ngôn ngữ, giao diện
      \item Phiên đăng nhập
    \end{itemize}
    \item Sau này được chuẩn hóa trong \textbf{RFC 2109} (1997) → \textbf{RFC 6265} (2011)
  \end{itemize}
\end{frame}

\begin{frame}{Sự ra đời của Cookie}
  \textbf{Cách hoạt động:}
  \begin{enumerate}
    \item Server gửi header: \texttt{Set-Cookie: user\_id=12345}
    \item Trình duyệt lưu cookie vào máy
    \item Trong các request sau, trình duyệt gửi: \texttt{Cookie: user\_id=12345}
  \end{enumerate}

  \vspace{0.5em}
  \textbf{Nhờ vậy:} Server có thể nhận biết user trong mỗi request
\end{frame}

\begin{frame}{Vấn đề bảo mật với Cookie}
  \begin{itemize}
    \item Cookie được lưu phía người dùng → người dùng có thể sửa!
    \item Ví dụ: \texttt{user\_id=12345} tự đổi thành \texttt{user\_id=1}
    \item Khi đó, người dùng có thể mạo danh người khác nếu server không kiểm tra cẩn thận
    \item Ngoài ra:
    \begin{itemize}
      \item Cookie có thể bị đánh cắp qua XSS
      \item Cookie gửi kèm mọi request → rò rỉ thông tin
    \end{itemize}
  \end{itemize}
\end{frame}

\begin{frame}{Giải pháp ra đời: Session!}
  \begin{itemize}
    \item Session bắt đầu xuất hiện và được chuẩn hoá khoảng từ năm 1996–1997
    \item Session lưu thông tin trên server, chỉ gửi ID cho client
    \item Cookie chỉ chứa \texttt{session\_id}, không chứa thông tin quan trọng
    \item Người dùng không thể sửa dữ liệu session vì nó nằm trên server
    \item Nhờ đó: bảo mật tốt hơn, tránh mạo danh và can thiệp
  \end{itemize}
\end{frame}

\begin{frame}{Tình huống thực tế 🤔}
\begin{itemize}
  \item Trên một trang web xem phim
  \item Tôi là tài khoản \textbf{free}, còn bạn là \textbf{premium}
  \item Tôi lấy được \texttt{cookie} của bạn (auth\_token, session\_id, v.v.)
  \item Tôi gán vào trình duyệt của mình và... \textbf{truy cập được phim premium!}
\end{itemize}

\vspace{1em}
\alert{Câu hỏi:} Liệu chỉ dùng session là đủ để ngăn chặn?
\end{frame}

\begin{frame}{Chỉ dùng Session liệu có đủ?}
\begin{itemize}
  \item Session lưu trên server – tốt hơn cookie thuần tuý
  \item Nhưng: session\_id vẫn gửi qua cookie → \textbf{có thể bị đánh cắp}
  \item Nếu attacker gán session\_id vào trình duyệt họ → Server vẫn tin tưởng!
\end{itemize}

\vspace{1em}
\textbf{Kết luận:} Session là cần thiết, nhưng \alert{không đủ} để ngăn giả mạo!
\end{frame}

\begin{frame}{Giải pháp chống giả mạo session}
\begin{itemize}
  \item 🔒 Ràng buộc session với:
  \begin{itemize}
    \item Địa chỉ IP gốc
    \item Trình duyệt (User-Agent)
    \item Device fingerprint (nếu cần)
  \end{itemize}

  \item ⏳ Hạn chế thời gian sống của session
  \item 🔁 Sử dụng access token ngắn hạn + refresh token dài hạn
  \item ✅ Luôn kiểm tra quyền truy cập phía server có được phép xem phim không, trước khi gửi file video về cho trình duyệt
\end{itemize}
\end{frame}

\begin{frame}{So sánh nhanh Cookie vs Session}
  \begin{tabular}{|l|c|c|}
    \hline
    \textbf{Tiêu chí} & \textbf{Cookie} & \textbf{Session} \\
    \hline
    Lưu ở đâu         & Trình duyệt     & Server \\
    \hline
    Bảo mật           & Thấp            & Cao \\
    \hline
    Dung lượng        & \textasciitilde 4KB & Tuỳ server \\
    \hline
    Thời gian sống    & Tuỳ cài đặt     & Tuỳ cấu hình \\
    \hline
  \end{tabular}
\end{frame}

\begin{frame}{Ứng dụng của Cookie}
  \begin{itemize}
    \item \textbf{Lưu tùy chọn người dùng}: Theme (tối/sáng), ngôn ngữ, layout cá nhân
    \item \textbf{Theo dõi giỏ hàng tạm thời}: Lưu ID sản phẩm trước khi người dùng đăng nhập
    \item \textbf{Theo dõi hành vi người dùng}: Dùng bởi Google Analytics, Facebook Pixel, v.v.
    \item \textbf{Ghi nhớ đăng nhập (Remember me)}: Duy trì trạng thái đăng nhập qua nhiều lần truy cập
    \item \textbf{Quản lý popup/thông báo}: Ghi nhớ người dùng đã tắt thông báo, không hiển thị lại
    \item \textbf{A/B Testing}: Gán user vào nhóm A hoặc B để thử nghiệm giao diện
  \end{itemize}
\end{frame}

\begin{frame}{Ứng dụng của Session}
  \begin{itemize}
    \item \textbf{Giỏ hàng trong e-commerce}: Lưu thông tin sản phẩm, số lượng theo session
    \item \textbf{Tạm lưu dữ liệu form}: Tránh mất dữ liệu nếu submit lỗi
    \item \textbf{Chống gửi form nhiều lần}: Xác thực token chống CSRF
    \item \textbf{Theo dõi user đang hoạt động}: Hiển thị trạng thái “online” trong hệ thống
    \item \textbf{Lưu vị trí điều hướng trước đó}: Redirect về trang trước sau khi đăng nhập
    \item \textbf{Lưu tiến trình thao tác}: Như trong các bước đăng ký nhiều bước (multi-step form)
  \end{itemize}
\end{frame}

\begin{frame}{Kết luận}
  \begin{itemize}
    \item Cookie \& Session là nền tảng cho mọi website hiện đại
    \item Từ chỗ ``không thể nhớ ai'' → Web cá nhân hóa, bảo mật hơn
    \item Giúp trải nghiệm người dùng trở nên mượt mà và đáng tin cậy
  \end{itemize}
\end{frame}
\end{document}
